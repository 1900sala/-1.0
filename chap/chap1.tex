% Copyright (c) 2008-2009 solvethis
% Copyright (c) 2010-2011 Casper Ti. Vector
% Public domain.

\chapter{引言}
	\section{研究背景}
	\indent 地震是地球上主要的自然灾害之一。每一次破坏性地震的发生都给人类社会带了巨大的人员财产损失。因此为了减小由于地震造成的巨大损失,人类通过对地震的学研究逐渐提高了对地震的认知和理解。为了更好的应对地震,就需要研究地震发生的机理即研究断层面上滑动的时空分布。目前,研究断层面上滑动的时空分布主要有两种方法:运动学方法和动力学方法。由于运动学研究不涉及导致产生断层面滑动的原因。因此,动力学方法是我们认识和了解地震发生的机制的主要研究手段。其中,断层自发破裂的传播问题是人们很感兴趣的一个震源动力学研究方向。但是由于问题的复杂性,通常在研究自发破裂问题的过程中难以得到闭合形式的解析解,因此数值模拟是人们研究该问题的重要手段。目前的研究中,三种主流的数值方法为:有限差分方法(FDM)、 有限元法(FEM) 和边界积分方程方法(BIEM)。 有限差分方法和有限元方法都是域方法。需要对区域划分大量的网格并求解大型的方程组,计算量大。而边界积分方程法通过给定空间下的格林函数,结合摩擦的本构关系求出了断层面上滑动时空分布,最终得到了自发破裂问题的解。对于一个三维空间的断层自发破裂问题,边界积分法实际处理的是一个二维问题。因此在该类问题数值模拟的过程中,边界积分方程法相对于另外两种方法具有更高的计算效率。\\
 \indent 边界积分方程法(~BIEM~)求解的出发点是位移表示定理(张海明,2004)。最初建立的是位移~BIEM,但由于该方法仅仅适用于无限空间中平面断层的模拟,因此为研究复杂断层系统的破裂过程,研究者建立了牵引力形式的~BIEM~方程(Tada,2000)。得到牵引力的边界积分方程后,需要对方程离散化以进行破裂过程的模拟。较早的方案是正方形离散化方案(Tada,2005),该方案适用于规则的平面断层。但是对于复杂断层,正方形网格并不能完全的铺满断层,因此在处理复杂断层的问题时,就显得不太适用。为解决这个问题,Tada(2006) 提出了能够模拟复杂断层系统的三角形离散化方案。基于~Tada(2000)~的结果,Feng and Zhang(2017)~用不同处理方式得到了相应的三角形离散化方案。对于~BIEM~的计算,时间复杂度为$O(M^{3}N)$,其中~M~为单元数,N~为时间步,因此在模拟一个大尺度模型时,需要高昂的计算成本。因此,优化~BIEM~的计算以减小时间复杂度是很有必要的。基于~Cochard and Madariaga~(1994)的结果,Lapusta~等人分别对~2-D(Lapusta \emph{et al},2000)~和~3-D(Lapusta and Liu ,2009)~的情形进行了计算的优化。而~Cochard and Madariaga~(1994)的结果只能应用于平面断层。对于非平面断层的情形,基于Tada(2006)的结果,Ando等人使用分区计算积分核的方法分别对~2D(Ando \emph{et al},2007)~和~3D(Ando,2016)~两种情形下~BIEM~的计算做了优化,降低了时间复杂度。

    \section{本文研究的意义}
\indent 本文的一个工作,是基于~Feng and Zhang(2017)~的三角形离散化方案,通过分区处理的方法优化离散化积分核积分的计算。积分核的计算是破裂过程数值模拟的重要环节,从物理意义上看,积分核表达的是各个离散单元之间的应力影响关系。在积分核的计算过程中,时间复杂度为$O(M^{3}N)$,其中~M~为单元数,N~为时间步。若是整个过程不采取任何优化方案,将产生巨大的时间开销,因此为了提高计算的效率我们需要采取一定的优化方案。为了优化积分核的计算过程,我们将积分核的计算划分为三个区域,初始区,传播区和稳定区。在初始区和稳定区积分核的形式和值都是固定的。对于这两个区域的积分核我们并不需要重复计算,只需单次计算便可得到整个区域中积分核的值,积分核计算的时间复杂度大大降低。因此通过分区的方法计算积分核可以极大的优化对于积分核的计算。\\
\indent 本文的另一部分工作是对破裂过程相关计算的优化。参与破裂过程的积分核的值为一个三维的稀疏矩阵,存在大量~0~元素与相同元素,造成了内存的浪费。因此,本文通过稀疏矩阵压缩存储技术实现了优化积分核存储空间的目的。但是,虽然达到了压缩矩阵的目的,但是目前的方法是以更大的计算成本为代价,需要更高的时间复杂度。这个地方的工作还需要进一步优化。因此,本文将矩阵压缩这个部分的内容放入到附录中,作为一个解决相关问题的参考和切入点。
\indent
