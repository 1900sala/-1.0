% Copyright (c) 2008-2009 solvethis
% Copyright (c) 2010-2011 Casper Ti. Vector
% Public domain.

\begin{cabstract}
\indent 地震预警是地震减灾工作的重要途径,而震级预估是整个地震紧急预警系统中重要且较为困难的一个环节。目前,世界上多个国家和地区都已建立了各自的地震预警系统,并且形成了特征频率($\tau_{p}$和$\tau_{c}$等)相关和特征振幅($P_{d}$等)相关的两类震级紧急预警的方法,但因各自算法出发点不同各有局限性。本文在已有的方法和理论基础上,运用机器学习类算法,将日本KIK和KNET台网从2015至2017年所记录到的840条地震目录,50314条记录作为全数据集。进行人工智能与震级预估领域的探索,并设计、训练出两套用于常见震级范围的机器学习震级预估模型。\\
\indent 与已有方法的预估结果相比,机器学习类方法不仅挺高了对地震事件预估震级的准确性和稳定性,表现出更低的误差和方差。同时多台联合评估地震事件的截面方差也更低,机器学习模型在只使用单台站记录进行震级预估时也显示出更优秀的可靠性。并且在面对低震级低信噪比事件或余震数据时,机器学习模型也展示出良好的抗噪和泛化能力。本研究的结果显示了机器学习算法在震级紧急预估问题上具有较广阔的应用前景,同时也为较为复杂的深度学习类算法框架下端到端模型提供了实践基础和研究可能。\\
\end{cabstract}

\begin{eabstract}
\indent Earthquake early warning is an important way for earthquake disaster reduction work, and magnitude prediction is an important and difficult part of the entire earthquake emergency warning system. At present, many countries and regions in the world have established their own earthquake early warning systems, and formed characteristic frequency ($\tau_{p}$ and $\tau_{c}$, etc.) correlation and characteristic amplitude ($P_{ d}$, etc.) There are two types of earthquake emergency warning methods, but they have their own limitations because of their different starting points. Based on the existing methods and theories, this paper uses the machine learning algorithm to record 840 earthquake catalogs and 50,314 records recorded by Japan's KIK and KNET networks from 2015 to 2017 as a full data set. Explore the field of artificial intelligence and magnitude estimation, and design and train two sets of machine learning magnitude prediction models for common magnitude ranges. \\
\indent Compared with the estimation results of the existing methods, the machine learning method not only improves the accuracy and stability of the earthquake event estimation magnitude, but also shows lower error and variance. At the same time, the cross-sectional variance of multiple joint assessment seismic events is also lower. The machine learning model also shows better reliability when using only single station records for magnitude estimation. And in the face of low-magnitude low SNR events or aftershock data, the machine learning model also shows good anti-noise and generalization capabilities. The results of this study show that the machine learning algorithm has a broad application prospect in the earthquake magnitude emergency estimation problem, and also provides a practical basis and research possibilities for the end-to-end model of the more complex deep learning algorithm framework.\\
\end{eabstract}

