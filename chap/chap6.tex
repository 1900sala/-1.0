        \appendix
\chapter{稀疏矩阵压缩}

\indent 参与破裂过程计算的积分核为稀疏矩阵,在存储的过程中会造成内存的浪费。针对积分核为稀疏矩阵的的情况,我们可以采取一定的矩阵压缩方案。

\section{基本思想}
\indent 在矩阵中,若数值为0的元素数目远远多于非0元素的数目,并且非0元素分布没有规律时,则称该矩阵为稀疏矩阵。因此,基本的想法是只记录稀疏矩阵非零元素的坐标和元素值,通过存储的坐标访问对应的元素值。对于找不到坐标信息的元素时,可直接得到该元素为~0~元素。一个~$4 \times 4$~的2维矩阵~B~为例
$$ B = 
\begin{bmatrix}
   0 & 2 & 0& 0 \\
   4 & 0 & 6 & 0\\
   0 & 8 & 0 & 0\\
   0 & 0 & 0 & 1
  \end{bmatrix} \tag{三维稀疏矩阵}
 $$
 我们记录矩阵B非零元素行列信息和对应的值如\ref{juzhen-yasuo}所示。这样原有的稀疏矩阵被拆分为两个部分存储,一个是非零元素的坐标信息,一个是非零元素的值,则存储空间为非零元素的两倍。假设零元素数目为~$m$~,非零元素为~$n$~。由于~$m>>n$~,因此 ~$2m<<m+n$。这我们实现了稀疏矩阵的压缩存储过程。

\begin{table}
\centering  
\caption{矩阵压缩方案} \label{juzhen-yasuo} 
\begin{tabular}{|c|c|c|c|}
 \hline
行坐标&列坐标& 矩阵值& 矩阵元素\\
 \hline
1 & 2 & 2& $B_{12}$ \\
 \hline
2&1 & 4 &  $B_{21}$\\
 \hline
2 & 3 & 6& $B_{23}$\\
 \hline
3 & 2 & 8& $B_{32}$ \\
 \hline
4 & 4 & 1& $B_{44}$ \\
\hline
\end{tabular}\\
\end{table}


 \section{存在的问题与反思}
 \indent 通过这样的矩阵压缩技术的确能够减小内存的占用,对于第三章中的数值算例一,在破裂过程计算中,积分核存储空间减少了~\%20,然而计算时间却增加了2倍之多。究其原因,我们可以发现,由于采取了压缩技术,在破裂过程的计算中需要添加判断条件来判断积分核的值,由于目前的破裂过程计算采用的是循环计算的方式,因此,时间累积,大大增加了我们的计算时间。因此,对于矩阵压缩,仍然由两个可以改进的方向:1、考虑积分核以矩阵的方式进行破裂过程的计算,这里需要对相应的矩阵计算进行优化。2、目前之考虑了~0~元素的情况,然而实际上,在非零元素中,处于稳定区的那部分稳定值也占据了不少的空间,因此对于稳定值存储的优化也是一个可以考虑的方向。
 
 
 \chapter{公式说明}
 
  在第二章中,公式\ref{con:L311}中各项的详细含义如下,$\Delta = CA-\frac{B^{2}}{4}$,$\Delta_{1}=EA-\frac{BD}{2}$,$\Delta_{2} = \Delta_{1} + (\frac{B^{2}}{2A}-C)F$,$r=(Ax^2+Bx+C)^{1/2} $,$r^{'} = \frac{Ax+\frac{B}{2}}{(Ax^2+Bx+C)^{1/2}}$。而不定积分~$f_{i}$~的表达式如下所示:
        \begin{align}
            f_{1}(A,B,C,D,E) &= \frac{1}{A} \frac{\Delta_{1}}{\Delta}r^{'}-\frac{D}{A} \frac{1}{r}
            \\
            f_{2}(A,B,C,D,E) &= \frac{\Delta_{1}}{\Delta}r^{'}(\frac{2}{3}\frac{1}{\Delta}+\frac{1}{3Ar^{2}})-\frac{D}{3A}\frac{1}{r^{3}}\\
            f_{3}(A,B,C,D,E) &=\frac{\Delta_{1}}{\Delta}r^{'}(\frac{8}{15}\frac{A}{\Delta^{2}}+\frac{4}{15\Delta r^{2}}+\frac{1}{5Ar^{4}})-\frac{D}{5A}\frac{1}{r^{5}} \\
            f_{4}(A,B,C,D,E) &= \frac{\Delta_{2}}{\Delta}r^{'}(\frac{2}{3}\frac{1}{\Delta}+\frac{1}{3Ar^{2}})-\frac{D-\frac{B}{A}F}{3A}\frac{1}{r^{3}}+\frac{F}{A}\frac{r^{'}}{\Delta}\\ \notag
            f_{5}(A,B,C,D,E) &= \frac{\Delta_{2}}{\Delta}r^{'}(\frac{8}{15}\frac{A}{\Delta^{2}}+\frac{4}{15\Delta r^{2}}+\frac{1}{5Ar^{4}})-\frac{D-\frac{B}{A}F}{5A}\frac{1}{r^{5}} + \frac{2}{3}\frac{F}{\Delta^{2}}r^{'} \\ 
            &+ \frac{F}{3A\Delta}\frac{r^{'}}{r^{2}}\\  \notag
        \end{align}
    \indent 此外,记定积分为公式\ref{con:dingjifen},想要确定~$s_1$~与~$s_2$,与它等价的问题是求当 ~$x \in [0,1]$~满足~f(x)<0~的范围。其中~f(x)~为一个二次函数,它的判别式为$\Delta$。 
    \begin{align} \label{con:dingjifen}
        \int_{0}^{1}H(t-r/c)dx=\int_{s_{1}}^{s_{2}} dx
    \end{align}
    
    当$\Delta >0$时,记方程~f(x)=0~的两根为~$r_1$~和~$r_2$~,分类讨论可以得到上下限满足:\\
    1.~$f(0) \le 0 $~,$~f(1)  \le 0$     ~~~~~~~$s_{1} =0$~,~$s_{2} = 1$\\
    2.~$f(0) \le 0 $~,$~f(1)  > 0$     ~~~~~~~$s_{1} =0$~,~$s_{2} = r_{2}$ \\
    3.~$f(0) > 0 $~,$~f(1)  \le 0$~~~~~~~~$s_{1} =r_{1}$~,~$s_{2} = 1$\\
    4.~$f(0) > 0 $~,$~f(1)  > 0$,$0<\displaystyle\frac{(b_{1}-a_{1})^{2} + (b_{2}-a_{2})^{2}}{(b_1 - a_1)(x_1 -b_1)+(b_2-a_2)(x_2-b_2)}<1$~(二次函数对称轴)~~~~~~~$s_{1} =r_{1}$~,~$s_{2} = r_{2}$\\
    5.~其它情况~~~~~~~原积分$  \int_{0}^{1}H(t-r/c)dx = 0$\\
    最后,为了避免混淆,我们约定
    \begin{align} 
         \int_{0}^{1}H(t-r/c_T)dx=\int_{z_{1}}^{z_{2}} dx \label{con:def-sxxian1}\\
          \int_{0}^{1}H(t-r/c_L)dx=\int_{y_{1}}^{y_{2}} dx \label{con:def-sxxian2}
    \end{align} 
    
