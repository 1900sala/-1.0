\chapter{总结与展望}
\indent 本文提出了使用机器学习类算法来解决和优化地震紧急预警系统中预估震级部分的思路,设计了两种神经网络模型并均完成了编写和实现。已实现的NN震级预估模型以多层感知机模型为基础的模型,与已有单台站震级预估算法相比有着更好的效果。通过较为简单的NN模型的实现展示出了此类算法在震级预估问题上的强大优势,同时也为深度学习类的模型探索出了今后研究方向。

\indent 从预估表现来看,与传统的$\mathrm{p}_{\mathrm{d}}$和$\tau_{c}$方法相比,NN震级预估模型无论是对于以地震事件为视角考察整体效果,还是以记录到同一事件的第一台站为视角,NN震级预估模型联所暴露出的方差和误差都更小,这意味着预估效果和稳定性都到提升。但根据预估震级结果可以分析出,只使用单台站进行震级预估的稳定性仍然有待提成,测试表明需要5个以上台站才能得到稳定的联合预估结果。

\indent 因计算机缓存和计算能力所限制而暂时没有能够完全使用全数据集的CNN震级预估模型,是以CNN卷积神经网络为基础深度学习模型。从已有研究结果和其思路来看,无论是使用高通巴特沃斯滤波器的$\mathrm{p}_{\mathrm{d}}$法、还是使用傅里叶变换的$\tau_{c}$法,本质都为对原始地震记录数据使用某种方式的滤波、变换后的某单一特征。这对于深度学习CNN网络表达能力是完全能够达到的,本文研究表明即使在使用更小数据集的情况下,CNN模型就已经展示出了优秀的震级预估能力。可以看出对于有一定深度CNN网络可期待模型通过其他的变换、滤波得到的未知特征。最后CNN模型可以将这些特征组合使用,得到更为优异的震级预估模型。

\indent NN模型中没有使用位置信息。对于正在改进中的CNN模型,可以考虑在网络最后卷积层Flatten后的全连接层上引入了台站和震源的经纬信息。在这种设置下模型如果只是用同一小区域的数据,在一定的时间尺度内我们可以假设当地地下结构和介质信息变化不大,使模型可能此在区域内获得断层信息、地球介质信息的记录能力。区别于像$\mathrm{p}_{\mathrm{d}}$方法单纯使用震中距对从台站获得的参量进行补偿处理。这可能是解决因各地地球介质差异而导致单一台站预估结果稳定性差的一种有效途径。\\

\indent 本文采取了在训练集中对大震级事件过采样的方法以应对因地震事件左偏的分布导致的训练样本分布不均。同时在第4章中也证明了相比于传统方法机器学习类模型对于低信噪比地震数据在预估震级领域有着更好的表现,这在一定程度上解决了传统方法无法预估大型地震后频繁发生的高危险性余震震级的问题。\\