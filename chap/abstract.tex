% Copyright (c) 2008-2009 solvethis
% Copyright (c) 2010-2011 Casper Ti. Vector
% Public domain.

\begin{cabstract}
\indent 地震预警是地震减灾工作的重要途径,而震级预估是整个地震紧急预警系统中重要且较为困难的一个环节。目前,世界上多个国家和地区都已建立了各自的地震预警系统,并且形成了特征频率(~$\tau_{p}$~和~$\tau_{c}$~等)相关和特征振幅($P_{d}$等)相关的两类震级紧急预警的方法,但各有局限性。本文在已有的方法和理论基础上,运用机器学习算法,将日本KIK和KNET台网从2015至2017年所记录到的843条地震目录,55426条记录作为全数据集,设计、训练出一套用于常见震级范围的机器学习震级预估模型。与已有方法的预估结果相比,机器学习方法不仅使预估的整体误差和方差下降,同时多台联合评估单一地震事件的截面方差也更低。本研究的结果显示了机器学习算法在震级紧急预估问题上具有较广阔的应用前景,同时也为较为复杂的深度学习类算法框架下端到端模型提供了实践基础和研究可能。
\end{cabstract}

\begin{eabstract}
\indent Earthquake early warning (EEW) is an important way for earthquake disaster reduction, and magnitude estimation is an important and difficult part of the entire EEW system. Nowadays, many countries and regions around the world have established their own EEW systems, and two types of magnitude emergency warning methods, characteristic frequency ((~$\tau_{p}$~and~$\tau_{c}$~, etc.) and characteristic amplitude ($P_{d}$ and others), have be presented. Based on the existing methods and theories, we applied the machine learning algorithm to 55,426 records for 843 earthquakes recorded by the KIK and KNET networks in Japan from 2015 to 2017. By using these records as a full data set, a set of machine learning magnitude prediction models have been designed and trained for common magnitude ranges. Compared with the estimated results of the existing methods, the machine learning method may reduce not only the estimated overall error and variance, but also the cross-sectional variance of multiple joint seismic events. The results of this study show that machine learning algorithm has a broad application prospect in earthquake magnitude emergency estimation, and provides a practical basis and research possibilities for end-to-end model of more complex deep learning algorithm framework as well.
\end{eabstract}

