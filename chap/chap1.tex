% Copyright (c) 2008-2009 solvethis
% Copyright (c) 2010-2011 Casper Ti. Vector
% Public domain.

\chapter{引言}
\section{研究背景}
\indent 我国是遭受地震灾害最为严重的国家之一。在过去的20世纪里,因大陆地震而造成死亡的人数中我国超过60万人,占全球总人数一半以上(金星等, 2003, 2012)。与此同时,随着近海陆地城市化进程的推进,我国60\%的百万人口城市都位于高烈度区域,其中包括全国23个省会城市,地震灾害对这些地区人民的生命和财产构成严重威胁(张红才等, 2012)。为了减少地震灾害造成的生命财产损失,地震预警 (EEW) 系统应运而生。它合理地利用城市和地震源之间的空间关系,同时配合人民群众接受适当的训练以响应地震警报信息,成为减少地震灾害的有效手段 (Allen et al., 2007)。\\
 \indent EEW系统对即将发生强烈震动的城市区域发出预警,在破坏性强的S波部分到达之前通常具有几秒到几十秒的预警时间,这个宝贵的时段可为各种关键设施预设紧急措施:例如紧急制动正在运行的高铁和动车,以避免潜在的脱轨;关闭天然气管道,以最大限度地减少火灾危险;紧急备份与关闭计算机等设施,以避免重要数据库丢失 (Kamigaichi et al., 2009; Allen et al., 2007; Iglesias et al., 2007)。EEW系统包括实时地震定位、实时震级预估、预警目标区烈度估计和预警信息发布四个主要部分。目前已经有成熟的方法进行地震定位和预警信息发布。但作为决定EEW系统好坏的重要环节的实时震级预估,仍存在着诸多难点:首先,大地震震级难以准确测定。较大的地震都是由多个断层破裂组成,单次断层破裂过程也十分复杂;其次,在EEW有时效性的要求下,短时窗内观测到信息少,常用的震级计算方法会在6.5级左右产生明显的震级饱和 (Kanamori et al., 2009);此外,发震断层走向和倾向等参数对地面运动分布也会产生显著影响,但这些震源参数难以实时得到。\\
 \indent 已有的解决方案在很多方面还有待提高。例如,在常见震级范围内漏报和误报情况出现频率较高;为了得到稳定的预估结果,需要的台站较多;仍无法解决震源参数高效实施获得等。单台站震级预估方法因其简单和快速的特点成为当前震级紧急预估的主流方法,但无论是以$\tau_{p}$和$\tau_{c}$ (Nakamura et al., 1993; Kanamori et al., 2005) 为核心的特征频率相关方法,还是以$\mathrm{P}_{\mathrm{d}}$(Wu et al., 2007; Kanamori et al., 2008, 2009) 为核心的振幅相关方法,都只使用了地震台站的垂直分量记录,导致信息极大的浪费。由于两类方法的出发点不同,各自具有一定的局限性,比如$\tau_{c}$类方法在实践中被证实在部分地区效果不好(Horiuchi et al., 2006),而$\mathrm{P}_{\mathrm{d}}$类方法对于长破裂时间的震源更容易震级饱和 (Zollo et al., 2006)。以上研究结果表明,以单一特征为核心的震级预估算法不够强健,促使了相对应的多特征间相容性的研究(张红才等, 2017)。另一方面,如果在EEW系统中加入实时震源参数检测,则会使得准确性与所用时间同时增加 (Heaton et al., 1985; Allen,2006)。例如,在意大利的Preto系统中,预先根据历史地震和对应断层预设数据库,发震后从数据库中自动匹配断层参数 (Weber et al., 2007),但在时效上仍难以达到实时预警的要求。\\
\section{本文研究的意义}
\indent 本文针对EEW系统的实时震级预估问题,采用机器学习类算法解决和优化以上提到的诸多问题。一方面,高密度地震台网记录的地震数据给需求巨量训练数据的机器学习算法提供了训练的可能性;另一方面,机器学习算法可以充分利用台站记录的完整数据,避免了已有算法中只利用单一分量造成的信息浪费。此外,考虑到不同地区断层有不同特点,机器学习算法能自动训练出最适合当地的震级预测模型,这使得模型可以获得最佳适应断层模型和震源参数的影响。本研究显示出新模型预估震级的能力显著优于已有方法,表明机器学习算法在震级紧急预估问题上具有较广阔的应用前景,同时也为较为复杂的深度学习类算法框架下端到端模型提供了实践基础和研究可能。\\
\indent 同时传统震级预估方法对于低信噪比数据的处理能力较低、抗噪声干扰差。以目前两类主流EEWs所使用的震级预估模块的算法特点,都还不能良好处理大型地震后频繁出现的余震。但事实上大型破坏地震之后的余震危害是巨大的,根据中国地震台网中心统计,2008年汶川地震主震发生后的三个月内,共发生6级以上的余震6次,5级至6级的余震30次。(陈九辉等,2009)。这些高强度余震很多时候都会被主震或其他与余震的尾波所干扰,导致使用传统方法在预估余震震级上往往会有较大的偏差。而机器学习CNN类算法在滤波抗噪方面有独特结构带来的优势,即使台站信息被噪音或主震尾波淹没仍可以在一定程度上,从中剥离出需要紧急预估的余震震级。\\
