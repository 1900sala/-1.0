% Copyright (c) 2008-2009 solvethis
% Copyright (c) 2010-2011 Casper Ti. Vector
% Public domain.

\chapter{引言}
\section{研究背景}
\indent 我国是遭受地震灾害最为严重的国家之一。在过去的20世纪里,因大陆地震而造成死亡的人数中我国超过60万人,占全球总人数一半以上(金星等, 2003, 2012)。与此同时,随着近海陆地城市化进程的推进,我国60\%的百万人口城市都位于高烈度区域,其中包括全国23个省会城市,地震灾害对这些地区人民的生命和财产构成严重威胁(张红才等, 2012)。为了减少地震灾害造成的生命财产损失,地震预警 (EEW) 系统应运而生。它合理地利用城市和地震源之间的空间关系,同时配合人民群众接受适当的训练以响应地震警报信息,成为减少地震灾害的有效手段 (Allen et al., 2007)。


 \indent EEW系统对即将发生强烈震动的城市区域发出预警,在破坏性强的S波部分到达之前通常具有几秒到几十秒的预警时间,这个宝贵的时段可为各种关键设施预设紧急措施:例如紧急制动正在运行的高铁和动车,以避免潜在的脱轨;关闭天然气管道,以最大限度地减少火灾危险;紧急备份与关闭计算机等设施,以避免重要数据库丢失 (Kamigaichi et al., 2009; Allen et al., 2007; Iglesias et al., 2007)。EEW系统包括实时地震定位、实时震级预估、预警目标区烈度估计和预警信息发布四个主要部分。目前已经有成熟的方法进行地震定位和预警信息发布。但作为决定EEW系统好坏的重要环节的实时震级预估,仍存在着诸多难点:首先,大地震震级难以准确测定。较大的地震都是由多个断层破裂组成,单次断层破裂过程也十分复杂;其次,在EEW有时效性的要求下,短时窗内观测到信息少,常用的震级计算方法会在6.5级左右产生明显的震级饱和 (Kanamori et al., 2009);此外,发震断层走向和倾向等参数对地面运动分布也会产生显著影响,但这些震源参数难以实时得到。

    \section{本文研究的意义}
\indent 本文的一个工作,是基于~Feng and Zhang(2017)~的三角形离散化方案,通过分区处理的方法优化离散化积分核积分的计算。积分核的计算是破裂过程数值模拟的重要环节,从物理意义上看,积分核表达的是各个离散单元之间的应力影响关系。在积分核的计算过程中,时间复杂度为$O(M^{3}N)$,其中~M~为单元数,N~为时间步。若是整个过程不采取任何优化方案,将产生巨大的时间开销,因此为了提高计算的效率我们需要采取一定的优化方案。为了优化积分核的计算过程,我们将积分核的计算划分为三个区域,初始区,传播区和稳定区。在初始区和稳定区积分核的形式和值都是固定的。对于这两个区域的积分核我们并不需要重复计算,只需单次计算便可得到整个区域中积分核的值,积分核计算的时间复杂度大大降低。因此通过分区的方法计算积分核可以极大的优化对于积分核的计算。\\
\indent 本文的另一部分工作是对破裂过程相关计算的优化。参与破裂过程的积分核的值为一个三维的稀疏矩阵,存在大量~0~元素与相同元素,造成了内存的浪费。因此,本文通过稀疏矩阵压缩存储技术实现了优化积分核存储空间的目的。但是,虽然达到了压缩矩阵的目的,但是目前的方法是以更大的计算成本为代价,需要更高的时间复杂度。这个地方的工作还需要进一步优化。因此,本文将矩阵压缩这个部分的内容放入到附录中,作为一个解决相关问题的参考和切入点。
\indent
