\chapter{总结与展望}
\indent 本文提出了使用机器学习类算法来解决和优化地震紧急预警系统中预估震级部分的思路,设计了两种神经网络模型并成功实现了其中一种。已实现的以NN网络为基础的模型,与已有单台站震级预估算法相比有着更好的效果。通过较为简单的NN模型的实现展示出了此类算法在震级预估问题上的强大优势,同时也为深度学习类的模型探索出了今后研究方向。

\indent 从预估表现来看,与传统的$\mathrm{p}_{\mathrm{d}}$和$\tau_{c}$方法相比,无论对于多事件整体,还是单一事件多台站联合预估NN模型的方差和误差都更小,这意味着预估效果得到提升。但根据单台站所得到的预估结果离散程度仍然很大,测试表明需要5个以上台站才能得到稳定的结果。

\indent 因数据量不足的原因暂时还未实现的CNN模型,是以CNN卷积神经结构为基础深度学习模型。从已有研究结果和其思路来看,无论是使用高通巴特沃斯滤波器的$\mathrm{p}_{\mathrm{d}}$法、还是使用傅里叶变换的$\tau_{c}$法,本质都为对原始地震记录数据使用某种方式的滤波、变换后的某一特征。这对于深度学习CNN网络表达能力是完全能够达到的,并且对于有一定深度CNN网络可期待其得到其他的变换、滤波下的未知特征。将这些特征组合使用,可望得到更为优异的模型。

\indent NN模型中没有使用位置信息,对于正在实现中的CNN模型,在其最后卷积层的输出后引入了台站和震源的经纬信息,这使其可能在区域内获得断层信息、地球介质信息的记录能力,而不是像$\mathrm{p}_{\mathrm{d}}$方法单纯使用震中距。这可能是解决单一台站预估结果离散过大的一种有效途径。

\indent 本文采取了在训练集中对大震级事件过采样的方法,以处理类别不平衡问题。此后还可以继续尝试使用再缩放法结合过采样法,进一步处理和优化天然地震震级分布不均的问题。



